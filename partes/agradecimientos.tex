%\documentclass[12pt,a4paper,twoside]{book} 
\usepackage[spanish]{babel} % de pedro
\usepackage{graphics,graphicx,epsfig,color,float,afterpage,fancyheadings,subfigure,moreverb,alltt} % de pedro
\usepackage[latin1]{inputenc} % tildes de pedro

\usepackage{algorithm}
\usepackage{algorithmic}

\usepackage{rotating}
\usepackage{url}

%% Esta letra se convierte mejor a pdf que la normal
\usepackage{ae}

%%% Para las fuentes matemticas
\usepackage{amsfonts}

\usepackage{subfigure}

\usepackage{pstricks} % para los dibujos del da
\usepackage{lscape} % para las pginas en horizontal
\usepackage{portland} % para las pginas en horizontal
\usepackage{supertabular} % para las tablas de ms de una pgina
\usepackage{tabularx} % para las tablas del tipo tabularx
%\usepackage{glossary}
%\documentclass[a4paper,spanish,12pt]{book} % esto es de gustavo
%\usepackage{amsmath,amsfonts}   % underset mathbb
%\usepackage{authordate1-4}      % bib style
%\usepackage{epsfig}     % eps
\usepackage{epic}           % graficos
%\usepackage{eepic}           % graficos
\usepackage{curvesls}           % curvas
\usepackage{amssymb}
%\usepackage{fancyheadings}  % encabezados
%\usepackage{hhline}             % hhline
%\usepackage[latin1]{inputenc}   % tildes
%\usepackage{makeidx}        % ndices
%\usepackage{setspace}           % interlinea
%\usepackage[spanish]{babel} % espaol

%%%%%%%%%%%%%%%%%%%%%%%%%%%%%%%%%%%%%%%%%%%%%%%%%%%%%%%%%%%%%%%%%%%%%%%%%%%%%%%

\author{juanlu}
\title{Tesis de Juan Lus Jimnez Laredo}




\newcommand{\fecha}{\footnotesize{[ Impreso: \the\day-\ifcase\month\or
    Ene\or Feb\or Mar\or Abr\or May\or Jun\or Jul\or Ago\or Sep\or
      Oct\or Nov\or Dic\fi-\the\year ]}}

\newcommand{\N}{\mathbb{N}}

%% Para corregir las cabeceras largas
\newcommand{\cabecera}[2]{
\markright{\ref{#1}. \hspace{0.1ex} \MakeUppercase{#2}}}


%\pagestyle{headings}
%\renewcommand{\chaptermark}[1]{\markboth{\fecha \\ \\ #1}{}}
%\renewcommand{\sectionmark}[1]{\markright{#1 \\ \\ \fecha}}
%\addtolength{\headheight}{2.5pt}



%\lhead[\it\thechapter]{\sl\rightmark}
%\rhead[\rm\leftmark]{\it\thesection}
%\rfoot[]{\thepage}
%\cfoot[]{}
%\lfoot[\thepage]{}

%\thispagestyle{plain}

\setcounter{secnumdepth}{3}
\setcounter{tocdepth}{3}

%\renewcommand{\baselinestretch}{1.2}
%\setlength{\parskip}{0.8ex}

\newtheorem{theorem}{\sf Teorema}
\newtheorem{lemma}{\sf Lema}

\newcommand{\rem}[1]{\S\iffalse #1 \fi}
\newcommand{\cur}[1]{ {\it #1\/} }
\newcommand{\crcl}[1]{#1\kern-9pt\raise1pt\hbox{$\bigcirc$}}
\newcommand{\evag}{{\sf EvAg}}
\newcommand{\evagp}{{\sf EvAg.}}
\newcommand{\evags}{{\sf EvAgs}}
\newcommand{\evagsp}{{\sf EvAgs.}}

\newcommand{\prog}[2] {
   \small
   \begin{minipage}[t]{75mm} {\tt #1}  \end{minipage}
   \begin{minipage}[t]{60mm} {#2}      \end{minipage}
   \\
}
\newcommand{\prg}[2] { {\tt #1} & {\sf #2} \\}

\newcommand{\wmfspecial}[4]{
   \begin{figure}[h]
   \centerline{\psfig{figure=#1,height=#2}}
   \caption{#3}   \label{#4}
   \end{figure}
}                   % USO: \wmfspecial{nombre.eps}{altura}{leyenda}{etiqueta}

\def\stackunder#1#2{\mathrel{\mathop{#2}\limits_{#1}}}

\def\marco #1#2#3#4{\centerline{       % USO: \marco{.1}{10}{124mm}
  \vbox{\hrule height #1pt%
  \hbox{\vrule width #1pt\kern #2pt%
  \vbox{\kern #2pt%
  \vbox{\hsize #3\noindent #4}%
  \kern #2pt}%
  \kern #2pt\vrule width #1pt}%
  \hrule height0pt depth #1pt}} }


\newcommand{\symnote}[2]{\symbolnote{#1}{#2}}

\newfont{\bi}{cmbxti10 scaled\magstep1}       % bf + it


%% Ruta de las figuras
\graphicspath{{../figuras/}}


\begin{document}


%%%%%%%%%%%%%%%%%%%%%%%%%%%%%%%%%%%%%%%%%%%%%%%%%%%%%%%%%%%%%%%%%%%%%%%%%%%%%%%
%%                                                                           %%
%%                                  Agradecimientos                          %%
%%                                                                           %%
%%%%%%%%%%%%%%%%%%%%%%%%%%%%%%%%%%%%%%%%%%%%%%%%%%%%%%%%%%%%%%%%%%%%%%%%%%%%%%%


\chapter* {Agradecimientos}

A mis directores de tesis, Juan Juli\'an Merelo y Pedro Castillo, principales responsables de que este escribiendo estas l\'ineas. Es a ellos a quienes se le deber\'ia de atribuir gran parte del posible m\'erito de este trabajo, con sus comentarios y correcciones han hecho que mejore en calidad y rigurosidad.


A todas aquellas personas que me han ayudado a amar a la ciencia y a entusiasmarme con ella, Ben Paechter por ofrecerme la oportunidad de compartir unos meses con \'el en Edimburgo, Gusz Eiben y Maarten van Steen por hacerme sentir que mi trabajo ten\'ia valor, Christian Gagn\'e por intercambiar ideas conmigo durante innumerables comidas en Laval. A Carlos Fernandes, mucho de lo bueno que pudiera tener esta tesis, se lo debo a \'el. A Marco Tomassini y Paulo Novais, por haberse ofrecido desinteresadamente a revisar esta memoria. A aquellos revisores que an\'onimamente se han tomado la molestia de estudiar mis art\'iculos, sus aportaciones han sido imprescindibles en mi proceso de maduraci\'on como investigador, alguno tal vez aparezca citado en la bibliograf\'ia de esta memoria, a estos autores tambi\'en, gracias, son los hombros en los que me he apoyado para ver un poco m\'as lejos. A los dos mejores divulgadores cient\'ificos que he tenido la suerte de conocer, Carl Sagan y Carlos Cotta. A todos mis compa\~neros de departamento, por intercambiar opiniones y momentos de desasosiego en numerosos desayunos, comidas, reuniones y tardes de f\'utbol. En especial, gracias a los miembros de geneura, esta tesis ha sido desarrollada gracias a su apoyo. 

A mis padres, porque son una lecci\'on viva de que la dignidad ni se compra ni se vende, adem\'as son unos cachondos mentales. A mi hermano Jos\'e, porque estaba s\'olo en el mundo hasta que lleg\'o \'el. A mi hermano Manolo, por su fortaleza de esp\'iritu y callado sacrificio, nunca te rompas, hay gente que naci\'o para ser grande (aunque sea criando gallinas) y nadie dijo que fuera f\'acil.

A mi abuelo Juan (Juan Cano) porque todav\'ia soy capaz de sentir la intensidad de su mirada. A mi abuela Josefa (Josefa la del loco) porque a\'un siento las caricias de sus manos. A mi abuelo Mariano (Luisillo el de los praos), la persona m\'as buena y sacrificada que conozco. A mi abuela Josefa (Josefa gui\~napa), porque llenas los espacios con tu encanto, guapa!

A mis t\'ios y primos, porque la mitad est\'an como cabras locas y me hacen sentir que pertenezco a la familia correcta, tito, a ver cuando me compro la moto.

A Lena, porque me hizo comprender que tanto en el para\'iso como en el infierno hay una temperatura de veintid\'os grados y sopla un poco de brisa. A todas las mujeres que he amado, porque me han ense\~nado a esperar con paciencia a la que voy a querer m\'as (mientras tanto aceptamos barco).

A mis amigos, innombrables y variopintos, gracias (Joder, Manu, nombraros a todos ser\'ia una pasada tio).

A mi pueblo, Zagra, por darme identidad en un mundo an\'onimo.

A mis ahijadas, Alba y Sheila, porque sois lindas.



\cleardoublepage


\thispagestyle{empty}
\bigskip
\bigskip
\bigskip
\bigskip
\begin{flushright}
{\em A mis padres y hermanos.}
\end{flushright}
\bigskip

%\end{document}