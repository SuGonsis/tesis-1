%\documentclass[12pt,a4paper,twoside]{book} 
\usepackage[spanish]{babel} % de pedro
\usepackage{graphics,graphicx,epsfig,color,float,afterpage,fancyheadings,subfigure,moreverb,alltt} % de pedro
\usepackage[latin1]{inputenc} % tildes de pedro

\usepackage{algorithm}
\usepackage{algorithmic}

\usepackage{rotating}
\usepackage{url}

%% Esta letra se convierte mejor a pdf que la normal
\usepackage{ae}

%%% Para las fuentes matemticas
\usepackage{amsfonts}

\usepackage{subfigure}

\usepackage{pstricks} % para los dibujos del da
\usepackage{lscape} % para las pginas en horizontal
\usepackage{portland} % para las pginas en horizontal
\usepackage{supertabular} % para las tablas de ms de una pgina
\usepackage{tabularx} % para las tablas del tipo tabularx
%\usepackage{glossary}
%\documentclass[a4paper,spanish,12pt]{book} % esto es de gustavo
%\usepackage{amsmath,amsfonts}   % underset mathbb
%\usepackage{authordate1-4}      % bib style
%\usepackage{epsfig}     % eps
\usepackage{epic}           % graficos
%\usepackage{eepic}           % graficos
\usepackage{curvesls}           % curvas
\usepackage{amssymb}
%\usepackage{fancyheadings}  % encabezados
%\usepackage{hhline}             % hhline
%\usepackage[latin1]{inputenc}   % tildes
%\usepackage{makeidx}        % ndices
%\usepackage{setspace}           % interlinea
%\usepackage[spanish]{babel} % espaol

%%%%%%%%%%%%%%%%%%%%%%%%%%%%%%%%%%%%%%%%%%%%%%%%%%%%%%%%%%%%%%%%%%%%%%%%%%%%%%%

\author{juanlu}
\title{Tesis de Juan Lus Jimnez Laredo}




\newcommand{\fecha}{\footnotesize{[ Impreso: \the\day-\ifcase\month\or
    Ene\or Feb\or Mar\or Abr\or May\or Jun\or Jul\or Ago\or Sep\or
      Oct\or Nov\or Dic\fi-\the\year ]}}

\newcommand{\N}{\mathbb{N}}

%% Para corregir las cabeceras largas
\newcommand{\cabecera}[2]{
\markright{\ref{#1}. \hspace{0.1ex} \MakeUppercase{#2}}}


%\pagestyle{headings}
%\renewcommand{\chaptermark}[1]{\markboth{\fecha \\ \\ #1}{}}
%\renewcommand{\sectionmark}[1]{\markright{#1 \\ \\ \fecha}}
%\addtolength{\headheight}{2.5pt}



%\lhead[\it\thechapter]{\sl\rightmark}
%\rhead[\rm\leftmark]{\it\thesection}
%\rfoot[]{\thepage}
%\cfoot[]{}
%\lfoot[\thepage]{}

%\thispagestyle{plain}

\setcounter{secnumdepth}{3}
\setcounter{tocdepth}{3}

%\renewcommand{\baselinestretch}{1.2}
%\setlength{\parskip}{0.8ex}

\newtheorem{theorem}{\sf Teorema}
\newtheorem{lemma}{\sf Lema}

\newcommand{\rem}[1]{\S\iffalse #1 \fi}
\newcommand{\cur}[1]{ {\it #1\/} }
\newcommand{\crcl}[1]{#1\kern-9pt\raise1pt\hbox{$\bigcirc$}}
\newcommand{\evag}{{\sf EvAg}}
\newcommand{\evagp}{{\sf EvAg.}}
\newcommand{\evags}{{\sf EvAgs}}
\newcommand{\evagsp}{{\sf EvAgs.}}

\newcommand{\prog}[2] {
   \small
   \begin{minipage}[t]{75mm} {\tt #1}  \end{minipage}
   \begin{minipage}[t]{60mm} {#2}      \end{minipage}
   \\
}
\newcommand{\prg}[2] { {\tt #1} & {\sf #2} \\}

\newcommand{\wmfspecial}[4]{
   \begin{figure}[h]
   \centerline{\psfig{figure=#1,height=#2}}
   \caption{#3}   \label{#4}
   \end{figure}
}                   % USO: \wmfspecial{nombre.eps}{altura}{leyenda}{etiqueta}

\def\stackunder#1#2{\mathrel{\mathop{#2}\limits_{#1}}}

\def\marco #1#2#3#4{\centerline{       % USO: \marco{.1}{10}{124mm}
  \vbox{\hrule height #1pt%
  \hbox{\vrule width #1pt\kern #2pt%
  \vbox{\kern #2pt%
  \vbox{\hsize #3\noindent #4}%
  \kern #2pt}%
  \kern #2pt\vrule width #1pt}%
  \hrule height0pt depth #1pt}} }


\newcommand{\symnote}[2]{\symbolnote{#1}{#2}}

\newfont{\bi}{cmbxti10 scaled\magstep1}       % bf + it


%% Ruta de las figuras
\graphicspath{{../figuras/}}


\begin{document}
           % Eliminarlo al compilar el documento maestro, ponerlo para compilarlo separado

%%%%%%%%%%%%%%%%%%%%%%%%%%%%%%%%%%%%%%%%%%%%%%%%%%%%%%%%%%%%%%%%%%%%%%%%%%%%%%%
%%                                                                           %%
%%                             Tesis Doctoral:                               %%
%%                        Juan Luis Jimenez Laredo                           %%
%%                                                                           %%
%%%%%%%%%%%%%%%%%%%%%%%%%%%%%%%%%%%%%%%%%%%%%%%%%%%%%%%%%%%%%%%%%%%%%%%%%%%%%%%

\cabecera{cap:conclusions}
                 {Conclusions}
\chapter{\textit{Conclusions}}
\label{cap:conclusions}
\cabecera{cap:conclusions}
                 {Conclusions}

%%%%%%%%%%%%%%%%%%%%%%%%%%%%%%%%%%%%%%%%%%%%%%%%%%%%%%%%%%%%%%%%%%%%%%%%%%%%%%%

This thesis studies the viability of the Peer-to-Peer Evolutionary Computation paradigm in the context of fine-grained spatially-structured Evolutionary Algorithms. To that end, the Evolvable Agent model has been presented and assessed empirically under different scenarios using additively-decomposable trap-functions as a benchmark.  Given that trap-functions have been designed to be difficult for Evolutionary Algorithms, the results should be easily extended to more general discrete or combinatorial optimization problems.


%The Evolvable Agent model defines a decentralised population structure by means of the gossiping protocol newscast that behaves asymptotically as a small-world graph. The influence of such kind of structures in the environmental selection pressure of Evolutionary Algorithms  is close to that in panmictic populations used by default in canonical approaches. Nevertheless, the inhomogeneities of small-world structured populations play an important role in the preservation of the genetic diversity, and have, therefore, a positive effect on scalability. Additionally, the approach shows to be robust under \emph{churn}; the departure of nodes does not inflict a penalisation in the execution time and the quality of solutions can be hold by simply increasing the initial population size.

In particular, this thesis provides the following contributions to the understanding of distributed Evolutionary Computation in Peer-to-Peer infrastructures:

\begin{itemize}

%%%%%%%%%%
\item[] {\bf The viability of the P2P EC paradigm has been analysed around the issues of decentralisation, scalability and fault-tolerance.} A main challenge on the design of P2P EAs is to provide a single coherent view of the system despite the nature of peers being decentralised. This way,  P2P systems  create powerful parallel infrastructures able to constitute a single virtual computer composed of a potentially large number of interconnected resources. In order to take full advantage of such systems, the motivation of a P2P EA is tackling large instances of difficult problems
at the increasing computing requirements that such instances claim. In that context of massive-scalability, failures of peers become inherent to the system and the approach has to demonstrate fault-tolerance in order to hold good performances.



%%%%%%%%%%
\item[] {\bf The \evag model is proposed as a simple and decentralised approach for P2P EC.} 
The Evolvable Agent model is a fine-grained spatially-structured EA that defines a decentralised population structure by means of the gossiping protocol newscast. This makes the approach suitable for a P2P execution in which every \evag can be potentially placed in a different peer. In that context, \evags evolve asynchronously in a loosely-coupled fashion with the mate selection locally restricted within their respective neighbourhoods.

%%%%%%%%%%
\item[] {\bf Linear speed-ups can be hold for large instances of demanding problems.}
In problems with very expensive fitness evaluation cost, the parallel performance of the \evag model has been shown to depend on the underlying computing platform. This way, for very demanding problem instances and high performance computer architectures, the approach has been estimated to hold linear speed-ups up to thousands of processors. In addition, the  multi-threading nature of the approach is able to take a seamless advantage of the different cores in desktop machines implementing a SMP architecture.
 
%%%%%%%%%%
\item[] {\bf The experimental analysis is based on the correct sizing of the populations.} 
Setting an adequate population size is a key to obtain good performances in EAs, that is, to preserve a good quality in the solutions without spending extra computational efforts. In order to estimate optimal population sizes, we have used an empirical method based on bisection. This way, investigating scalability is possible by observing how population sizes and computational efforts scale for different problem instances.

%%%%%%%%%%
\item[] {\bf The source code of the simulator is released as open-source.} 
We find that practitioners can benefit from it either reproducing experiments or extending the framework. The simulator can be found at the subversion repository \url{https://forja.rediris.es/svn/geneura/evogen}, published under GPL v3 license.

%%%%%%%%%%
\item[] {\bf P2P EAs scale well for increasing problem sizes and difficulties.}
\evag has been shown to scale better than canonical approaches requiring of a smaller population size and number of evaluations as problem instances become large. The improvement is much more visible as the problem difficulty increases showing the adequacy of the P2P approach for tackling large instances of difficult problems. In addition, the comparison against a ring structured population shows that such a regular lattice population requires of a larger number of evaluations having therefore, a worst algorithmic performance.

%%%%%%%%%%
\item[] {\bf Small-world population structures offer good trade-offs between exploitative and explorative components of EAs.} 
The inhomogeneities of small-world structured populations play an important role in the preservation of the genetic diversity and have a positive effect on scalability. On the one hand, the influence of the environmental selection pressure is not so high as in panmictic populations so that population sizes and times to solutions can be reduced. In addition, the progress of the genetic diversity points out a more exploitative behaviour than the one induced by regular lattices. This way, the number of evaluations required to find optimal solutions can be minimised. 

In that context, we have compared the performance of the Watts-Strogatz and newscast methods generating small-world population structures. Both methods have been shown to yield similar results outperforming, in turn, the algorithmic scalability of the rest of approaches in this thesis.

%%%%%%%%%%
\item[] {\bf P2P EAs suffer a graceful degradation under churn.}
The \evag approach has been shown to be robust under different  degradation rates of \emph{churn}; the departure of nodes does not inflict a penalisation in the execution time and the quality of solutions can be hold by simply increasing the initial population size.

\end{itemize}


We find that previous contributions represent a great advance to a better understanding of the P2P EC paradigm by pointing out the main issues and showing the viability of the approach. Nevertheless, there is much work to be done for a whole comprehension of P2P EAs and, more in general, parallel EAs.   In that line,  we plan to focus in future works on the following challenges that have been identified throughout the development of the thesis:

\begin{itemize}
%%%%%%%%%%
\item[] {\bf Validation of the model in a real P2P infrastructure:} 
After studying the viability of the \evag model, the validation of the approach is a logical further step. To that aim, we plan to deploy the algorithm in a real P2P platform for tackling demanding problem sets such as the very large instances of the vehicle routing problem \cite{vlvrp}. This way, it will be possible to focus on engineering issues such as the real impact of the latency and bandwidth on the algorithm performance or the exploration of load-balancing methods to cope with the issue of heterogeneity in peers.

%%%%%%%%%%
\item[] {\bf Exploration of other P2P protocols as population structures:}
This thesis has shown that different methods generating small-world population structures have equivalent performances in EAs. This fact is remarkable since there are many P2P protocols designed to work as small-world networks as Gnutella 0.4 \cite{gnutella04} or any DHT \cite{chord,pastry,tapestry,can}.
Therefore, we aim to explore some of these P2P protocols as population structures for EAs so that P2P EC can benefit from  existing P2P platforms.


%%%%%%%%%%
\item[]{\bf Extension of the P2P concept to other optimisation meta-heuristics:} 
To this point, we have focused in P2P optimisation within the concrete field of EC, however, there are no restrictions imposing a limitation of the P2P approach to other paradigms as they could be EDAs, ACO or PSO. In fact, PSO has already received some attention in the literature, e.g. the P2P-MOPSO approach by Scriben et al. in \cite{scriven:p2ppso} or the one by B\'anhelyi et al. in \cite{balazs:evo09}.

%%%%%%%%%%
%\item[] {\bf Applications to Multi-objective problems:}
%Aggregative functions on the estimates of the pareto-front

%%%%%%%%%%
%\item[] {\bf Further development of the simulator:}
%At this point, most of the components in the simulator are under a phase of development.


%%%%%%%%%%
%\item[] {\bf Generalisation of a methodology for general developments in P2P systems:}

%Stage 0. Analyse the main issue involving the problem 

%Stage 1. Viability Analysis.
%It would consist in the fast prototyping of aggregative solutions for a given problem using simulations (PeerSim, EvoGen,....). (This would
%include as well data-mining investigations) (A possible adventage here could be to obtain earlier and easier results
%than in developments for real infrastructures).

%Stage 2. Validation Analysis.
%This would consist in a slower stage for the development and testing of the prototype in real infrastructures (such as DAS cluster).

%Stage 3. Refinement of the prototype in Stage 1 to meet the new issues involving Stage 2.

%Stage 4. Deployment in ad-hoc P2P infrastructure.

\end{itemize}


As a final consideration and despite some future lines of work have been exposed, we feel that P2P optimisation is on its beginnings yet and this thesis can be a source of ideas and motivation for practitioners and theoreticians to do further investigations in the area.

\clearpage    
\section{Published papers related to the thesis}    
    
 During the development of this thesis, and directly related to it, the following papers were published on different peer reviewed journals and conference proceedings:

\begin{description}

\item[Peer reviewed journal papers]:

\begin{enumerate}

\item Juan Luis Jimenez Laredo, Agoston E. Eiben, Maarten van Steen, and Juan Julian Merelo. Evag: A
    scalable peer-to-peer evolutionary algorithm. Genetic Programming and
     Evolvable Machines, 2010. http://dx.doi.org/10.1007/s10710-009-9096-z.

\item Juan Luis Jimenez Laredo, Pedro A. Castillo, Antonio M. Mora, Juan Julian Merelo, and Carlos Fernandes.
    Resilience to churn of a peer-to-peer evolutionary algorithm. Int. J. High Performance Systems Architecture, 1(4):260-268, 2009.


\item Juan Luis Jimenez Laredo, Pedro A. Castillo, Antonio Miguel Mora, and
     Juan Julian Merelo Guervos. Evolvable agents, a fine grained approach for distributed evolutionary computing: walking towards the peer-to-peer computing frontiers. Soft Computing - A Fusion of Foundations,
     Methodologies and Applications, 12(12):1145-1156, 2008.

\end{enumerate}

\item[Peer reviewed conference papers and book chapters]:

\begin{enumerate}

\item Juan Luis Jimenez Laredo, Juan Julian Merelo Guervos, and Pedro Angel Castillo Valdivieso. Paral. and Distrib. Comp. Intel., volume 269 of SCI, chapter Evolvable Agents: A Framework for Peer-to-Peer Evolutionary Algorithms, pages 43-62. Springer-Verlag Berlin Heidelberg, 2010.


\item Juan Luis Jimenez Laredo, Carlos Fernandes, Juan Julian Merelo, and Christian Gagne. Improving genetic algorithms performance via deterministic population shrinkage. In GECCO' 09: Proceedings of the 11th Annual
     conference on Genetic and evolutionary computation, pages 819-826, New York, NY, USA, ACM, 2009.

\item Juan Luis Jimenez Laredo, Carlos Fernandes, Antonio Mora, Pedro A. Castillo, Pablo Garcia-Sanchez, and Juan Julian Merelo. Studying the cache size in a gossip-based evolutionary algorithm. In G.A. Papadopoulos
     and C. Badica, editors, Proceedings of the 3rd International Symposium on Intelligent Distributed Computing, volume 237 of Studies in Computational Intelligence, pages 131-140. Springer-Verlag Berlin Heidelberg,
     2009.

\item Juan Luis Jimenez Laredo, Pedro A. Castillo, Antonio M. Mora, Carlos Fernandes, and Juan Julian Merelo. Addressing Churn in a Peer-to-Peer Evolutionary Algorithm. In Juan Lanchares, Francisco Fernandez, and Jose L. Risco-Martin, editors, WPABA'  08 - First International Workshop on Parallel Architectures and Bioinspired Algorithms, Toronto, Canada, pages 5-12. Complutense University Of Madrid, 2008.



     
\item Juan Luis Jimenez Laredo, Pedro A. Castillo, Antonio Miguel Mora,
     Juan Julian Merelo Guervos, Agostinho Claudio da Rosa, and Carlos
     Fernandes. Evolvable agents in static and dynamic optimization problems. In Rudolph et al., editors. Parallel Problem Solving from Nature - PPSN X,
     10th International Conference Dortmund, Germany, Proceedings, volume 5199 of Lecture Notes in Computer Science, pages 488-497. Springer, 2008

\item Juan Luis Jimenez Laredo, Agoston E. Eiben, Maarten van Steen, and
     Juan Julian Merelo Guervos. On the run-time dynamics of a peer-to-peer
     evolutionary algorithm. In Rudolph et al., editors. Parallel Problem Solving from Nature - PPSN X,
     10th International Conference Dortmund, Germany, Proceedings, volume 5199 of Lecture Notes in Computer Science, pages 236-245. Springer, 2008.


\item Juan Luis Jimenez Laredo, Agoston E. Eiben, Maarten van Steen, Pedro A.
     Castillo, Antonio Miguel Mora, and Juan Julian Merelo Guervos. P2P
     evolutionary algorithms: A suitable approach for tackling large instances
     in hard optimization problems. In Emilio Luque et al., editors, Euro-Par, volume 5168 of Lecture Notes in
     Computer Science, pages 622-631. Springer, 2008.

\item Juan Luis Jimenez Laredo, Pedro A. Castillo, Antonio M. Mora, and Juan Julian Merelo. Exploring  population structures for locally concurrent and massively parallel evolutionary algorithms. In IEEE Congress on Evolutionary Computation (CEC2008), WCCI2008 Proceedings, pages 2610-2617. IEEE Press, Hong Kong, June 2008.


\item Agoston E. Eiben, Marc Schoenauer, Juan Luis Jimenez Laredo, Pedro A. Castillo Valdivieso, Antonio Miguel Mora, and Juan Julian Merelo. Exploring selection mechanisms for an agent-based distributed evolutionary algorithm. In Dirk Thierens, editor, GECCO' 07, pages 2801-2808. ACM, 2007.

     
\item Juan Luis Jimenez Laredo, Pedro A. Castillo Valdivieso, Ben Paechter, Antonio Miguel Mora, Eva Alfaro-Cid, Anna Esparcia-Alcazar, and  Juan Julian Merelo Guervos. Empirical validation of a gossiping communication mechanism for parallel EAs. In Mario Giacobini et al., editors, EvoWorkshops, volume 4448 of Lecture Notes in Computer Science, pages 129-136. Springer, 2007.

\item Juan Luis Jimenez Laredo, Pedro A. Castillo, Antonio M. Mora, and Juan Julian Merelo. Escalado con un sistema de agentes evolutivos distribuido. In Francisco Almeida-Rodriguez et al., editors, Actas MAEB 2007, pages 111-118, 2007.
    
\item Juan Luis Jimenez Laredo, Pedro A. Castillo, Gustavo Romero, Antonio M. Mora, Juan Julian Merelo, and
    Maribel G. Arenas. Validacion de un sistema computacional P2P mediante un estudio empirico. In XVII Jornadas de Paralelismo - XVII JP, pages 235-240, September 2006.
    
\item Juan Luis Jimenez Laredo, Pedro A. Castillo, Antonio M. Mora, and Juan Julian Merelo. Estudio preliminar sobre autoadaptacion en agentes evolutivos sobre arquitecturas heterogeneas. In XVII Jornadas de Paralelismo - XVII JP, pages 389-394, September 2006.

\end{enumerate}
\end{description}



%\bibliographystyle{plain}  % Eliminarlo al compilar el documento maestro, ponerlo para compilarlo separado
%\bibliography{evagperformance}%evagperformance,pea,p2pcomputing,model,methodology} %Eliminarlo al compilar el documento maestro, ponerlo para compilarlo separado

%\end{document}             % Eliminarlo al compilar el documento maestro, ponerlo para compilarlo separado




 
