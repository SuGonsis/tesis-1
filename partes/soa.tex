%%%%%%%%%%%%%%%%%%%%%%%%%%%%%%%%%%%%%%%%%%%%%%%%%%%%%%%%%%%%%%%%%%%%%%%%%%%%%%%
%%                                                                           %%
%%                             Trabajo indito                               %%
%%                                                                           %%
%%%%%%%%%%%%%%%%%%%%%%%%%%%%%%%%%%%%%%%%%%%%%%%%%%%%%%%%%%%%%%%%%%%%%%%%%%%%%%%

\cabecera{cap:trabajo}
                 {State of the art}
\chapter{\textit{State of the art}}
\label{cap:trabajo}
\cabecera{cap:trabajo}
                 {State of the art}

%%%%%%%%%%%%%%%%%%%%%%%%%%%%%%%%%%%%%%%%%%%%%%%%%%%%%%%%%%%%%%%%%%%%%%%%%%%%%%%

%\vfill
%\itshape
%En los problemas de clasificacin de patrones se busca minimizar el nmero de patrones mal clasificados (el error), sin embargo, en muchas aplicaciones reales hay que tener en cuenta por separado el error tipo I (falsos positivos) y el error tipo II (falsos negativos).
%Suele ser un problema complejo ya que un intento de minimizar uno de ellos, hace que el otro crezca.
%Es ms, en ocasiones uno de estos tipos de error puede ser ms importante que el otro, y se debe buscar un compromiso que minimice el ms importante de los dos.
%La medida estadstica ms utilizada, significancia estadstica, es una medida del error tipo I. Sin embargo, no ofrece garantas sobre el tipo II.

%A pesar de la importancia de los errores tipo II, la mayora de los mtodos de clasificacin slo tienen en cuenta el error de clasificacin global.
%En este trabajo se propone la optimizacin de ambos tipos de error de clasificacin utilizando un algoritmo multiobjetivo en el que cada tipo de error y el tamao de red son objetivos de la funcin de evaluacin (fitness).

%Se ha utilizado una versin modificada del mtodo G-Prop (diseo y optimizacin de perceptrones multicapa usando un algoritmo evolutivo) para optimizar simultneamente la estructura de la red neuronal y los errores tipo I y II.

%Debido a la carga computacional que supone la ejecucin de un algoritmo evolutivo para el diseo de redes neuronales, se propone la paralelizacin utilizando el modelo isla como forma de distribuir la carga en una red heterognea.
%\upshape
%\clearpage

%%%%%%%%%%%%%%%%%%%%%%%%%%%%%%%%%%%%%%%%%%%%%%%%%%%%%%%%%%%%%%%%%%%%%%%%%%%%%%%

%%%%%%%%%%%%%%%%%%%%%%%%%%%%%%%%%%%%%%%%%%%%%%%%%%%%%%%%%%%%%


The idea of distributed Evolutionary Algorithms was proposed quite
early (e.g. Grefenstette in \cite{grefenstette}); nowadays, parallel
EAs are approached mainly under three methodologies: 
 master-slave, islands and fine grained spatially structured
 EAs. However, P2P EAs are more recent and not all the models fit with
 the issues involving P2P systems, such as decentralization, massive
 scalability or fault tolerance, as we will see below.
% Podría ponerse con un \begin{itemize}, para que se viera que estás
% examinando los diversos paradigmas y su relación con los sistemas
% P2P? - JJ

\begin{itemize}

\item In the master-slave mode the
algorithm runs on the master node and the individuals are sent for evaluation
to the slaves, in an approach usually called {\em farming}. Such an architecture does not match decentralized structures and the master represents a single point of failure. 

\item One of the most usual and widely studied approaches in parallel EAs is the Island model (see \cite{	cantu:parallelga} for a survey). The idea behind this model is that the global panmictic population is split in several sub-populations or demes called islands. The communication pipes between islands are defined by a given
topology, through which they exchange individuals (migrants) with a certain rate and frequency.   The migration will follow a selection policy in the source island and a replacement policy in the target one.
Practitioners use to establish a fixed population size $P$ in scale-up studies, a number of islands $N$
and a population size per island of $P/n$ where $n = 1,\dots,N$. 
The work by Hidalgo and Fern\'andez \cite{Fernandez:balancing} requires a special attention. They experimentally show how the algorithmic results are strongly dependent on the number of islands. Our experimentation in \cite{laredo2008:softcomputing} with the Island model is consistent with such a conclusion since it shows  to be very sensitive to parameter calibration and P2P systems do not provide a priori knowledge of the global environment that an island model would need in order to set parameters such as the number of islands, the population size per island and the migration rate. 

\item Most of the works regarding finer grained approaches for parallel EAs focus on the algorithmic effects of using different topologies  i.e. 
Giacobini et al. study the impact of different neighbourhood structures on the selection pressure in regular lattices \cite{giacobini:regular} and different graph structures such as a toroid \cite{giacobini:gecco04} or small-world \cite{giacobini:gecco05}. This last structure has shown empirically to be competitive against panmictic EAs \cite{preuss04ppsn, giacobini:evocop06}. Fine grained approaches are more suitable for decentralization as stated in \cite{upali:adaptive,laredo:selection}, where
the key underlying idea is that individuals evolve within a defined set of neighbours. Following this line, we presented in \cite{laredo:cec2008} a formal model for P2P EAs, it is the \emph{Evolvable Agent model} that we analyse in this chapter. The model uses the gossiping protocol newscast \cite{jelasity:newscast} as population structure. Newscast was proposed within the DREAM project \cite{arenas:dream}, one of the pioneers in distributed P2P EAs. Although, the island-based parallelization of DREAM was shown in \cite{laredo:empirical} to be insufficient for tackling large-scale decentralized scenarios. 
\end{itemize}

On the other hand, protocols such as newscast have been designed taking
into account both EAs and P2P architecture: as a gossiping protocol, newscast is scalable and robust \cite{spyros:robustscalable}. There are evidences that such properties extend to P2P EAs using newscast as population structure. The scalability analysis in \cite{laredo08:large} shows that the \emph{EvAg} model is able to tackle large instances of hard optimization problems in reasonable time (i.e. the runtime scales with fractional power with respect to the problem instance). Additionally, the study of fault tolerance in \cite{laredo08:churn} shows that, whenever a P2P system guarantees enough peers at the beginning of an experiment, the departure of nodes 
does not inflict a penalization either on the convergence nor on the
% después de not se usa either - JJ
runtime. Such a study uses the model of \emph{churn} proposed by
Stutzbach and Rejaie in \cite{Stutzbach06Understanding}. Indeed,
Fern\'andez de la Vega et al. in \cite{vega06:faulttolerance} advanced that
Evolutionary Algorithms are fault tolerant because of its nature and
design.  
