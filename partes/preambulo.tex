%%%%%%%%%%%%%%%%%%%%%%%%%%%%%%%%%%%%%%%%%%%%%%%%%%%%%%%%%%%%%%%%%%%%%%%%%%%%%%%
%%                                                                           %%
%%              Preambulo: Definicion del formato del proyecto               %%
%%                                                                           %%
%%%%%%%%%%%%%%%%%%%%%%%%%%%%%%%%%%%%%%%%%%%%%%%%%%%%%%%%%%%%%%%%%%%%%%%%%%%%%%%

%%% Paquetes para el castellano.
%\usepackage[T1]{fontenc}
%\usepackage[latin1]{inputenc}
%\usepackage[spanish]{babel}

%%% Para utilizar el entorno de Pseudocódigo

\usepackage{algorithm}
\usepackage{algorithmic}
%% Para incluir graficos
%\usepackage{graphicx}        % standard LaTeX graphics tool
\usepackage{subfigure}

\usepackage{rotating}
\usepackage{appendix}

%% Esta letra se convierte mejor a pdf que la normal
\usepackage{ae}

%%% Para las fuentes matemticas
\usepackage{amsfonts}
\usepackage{amssymb}

%%% Para hypervnculos. Compila tanto con pdflatex como con latex
%\newif\ifPDF
%\ifx\pdfoutput\undefined\PDFfalse
%\else\ifnum\pdfoutput >0\PDFtrue
%    \else\PDFfalse
%    \fi
%\fi
 
%\ifPDF
\usepackage{graphicx}
%\usepackage{pstricks} % para los dibujos del da
\usepackage{epic}           % graficos
%\usepackage{curvesls}           % curvas
% \usepackage[dvips=false,pdftex=false,vtex=false]{geometry}

\usepackage{afterpage,fancyheadings}

%\usepackage[cam,b5,center,dvips]{crop}
 
    \usepackage[pdftex]{color}
    \usepackage[pdftex]{hyperref}
%\else
%    \usepackage[dvips]{graphicx,color}
%    \usepackage[dvips]{hyperref}
%\fi
 \usepackage{ccaption}

  % Change the format of a figure caption
  % For more options see the package documentation
  \captionnamefont{\bfseries}
  \captiontitlefont{\small\sffamily}
  \captiondelim{ --- }
  \hangcaption
  \renewcommand{\figurename}{Fig.}


%% Configuracin de hyperref
\hypersetup{%
colorlinks=true,
linkcolor=black,
citecolor=black,
urlcolor=blue,
bookmarks=true,
bookmarksopen=false,
hypertexnames=false,
breaklinks=true,
pdftitle={Tesis doctoral},
pdfauthor={Juan Luis Jimenez Laredo},
%pdfkeywords={pdf, latex, tex, ps2pdf, dvipdfm, pdflatex},
bookmarksnumbered,
pdfstartview={FitH}
}%


%% Ruta de las figuras
\graphicspath{{figuras/}}

%%% Para la apariencia de las pginas
%\pagestyle{headings}
%\setcounter{secnumdepth}{10}
%\setcounter{tocdepth}{10}
%\renewcommand{\baselinestretch}{1.3}
%\setlength{\parskip}{2ex}

%%% Modificacion de los estilos de página

%\pagestyle{fancy}
%  \def\headrulewidth{0.4pt}
%  \def\footrulewidth{0.4pt}
%  \renewcommand{\chaptermark}[1]{\markboth{#1}{}}
%  \renewcommand{\sectionmark}[1]{\markright{#1}}
%  \addtolength{\headheight}{2.5pt}
%    \lhead[]{}
%    \rhead[]{}
%    \rfoot[]{\thepage}
%   \cfoot[]{}
%    \lfoot[\thepage]{}
%  \setcounter{secnumdepth}{3}
%  \setcounter{tocdepth}{3}


%\lhead[\it\thechapter]{\sl\rightmark}
%\rhead[\rm\leftmark]{\it\thesection} \rfoot[]{\thepage} \cfoot[]{}
%\lfoot[\thepage]{} \pagenumbering{arabic}



%% Para el smbolo del euro
\usepackage{eurofont}

%%% Configuramos para un tamao de papel A4
%\setlength{\paperwidth}{21cm}
%\setlength{\paperheight}{29.7cm}

%\setlength{\textwidth}{14.5cm}
%\setlength{\textheight}{23.5cm}
%\setlength{\oddsidemargin}{1.4cm}
%\setlength{\evensidemargin}{0.2cm}
%\setlength{\topmargin}{-0.2cm}

%%% Para que los vectores salgan sin flechas
\def\vec#1{\mathchoice{\mbox{\boldmath$\displaystyle#1$}}
{\mbox{\boldmath$\textstyle#1$}}
{\mbox{\boldmath$\scriptstyle#1$}}
{\mbox{\boldmath$\scriptscriptstyle#1$}}}

%% Para corregir las cabeceras largas
\newcommand{\cabecera}[2]{
\markright{\ref{#1}. \hspace{0.1ex} \MakeUppercase{#2}}}

%% Para cambiar el formato del texto de las urls
\newcommand{\texturl}[1]{\textsf{#1}}

%% Para crear urls
\newcommand{\miurl}[1]{\href{#1}{\texturl{#1}}}

%% Para cambiar el formato del texto de las tablas
\newcommand{\formatotabla}[0]{\sf \footnotesize \renewcommand{\baselinestretch}{1.1}}

\newcommand{\N}{\mathbb{N}}
%%% Para que las tablas sean tablas en vez de cuadros
%\renewcommand\listtablename{ndice de tablas}
%\renewcommand\tablename{Tabla}

\newcommand{\evag}{{\sf EvAg }}
\newcommand{\evags}{{\sf EvAgs }}
\newcommand{\evagp}{{\sf EvAg. }}
\newcommand{\evagsp}{{\sf EvAgs. }}
\newcommand{\evagc}{{\sf EvAg, }}
\newcommand{\evagsc}{{\sf EvAgs, }}



%%% Para que parta ``bien'' algunas palabras
\hyphenation {hard-ware soft-ware ins-truc-cio-nes de-sa-rro-lla-ron mo-de-lo mi-ni-mi-zar res-tric-cio-nes pa-ra-me-tros pro-ba-bi-li-dad ge-ne-ra-li-zar ge-ne-ra-li-za-cion pre-via-men-te}

%%%%%%%%%%%%%%%%%%%%%%%%%%%%%%%%%%%%%%%%%%%%%%%%%%%%%%%%%%%%%%%%%%%%%%%%%%%%%%%
