\myChapter{Evolutionary Algorithms}\label{chap:introduction}
\minitoc\mtcskip
\vfill

%%% IMPORTANTE: JJ NO REVISES ESTO AUN!!!!!!!!!

\lettrine{E}{volutionary}

\section{Motivation of the thesis}

Open Science

Interoperability

Dynamism 

Therefore, the objective of the thesis is to demonstrate that SOA is the , also answering the next questions:
\begin{itemize}
\item Which is the best paradigm to expose EAs as Open Science? (CAMBIAAAAA)
\item What advantages can offer SOA to EA researchers?
\item 
\end{itemize}

The applications...

\begin{itemize}
\end{itemize}


\section{Structure of the thesis}

Current chapter shows an introduction to this thesis, with the motivations and question to address. 

Chapter \ref{chap:distributed} describes the traditional classification of the EAs, showing that EAs follow a number of common steps that can be recombined to create new algorithms or being used dynamically. Also, new trends in distributed EAs, such as P2P or pool-based algorithms that deals with heterogeneous architectures and dynamism with the nodes. This chapter also shows different frameworks for EAs using different programming languages, but without any mechanism to facilitate the integration, being the lack of standardization a problem (as suggested by \person{Parejo \etal} in \{SURVEYMOFS}).

In chapter \ref{chap:soa} the Service Oriented Architecture paradigm is presented as a possible solution to deal with the issues described in previous chapter, because offers mechanisms to standardize and open science... Different technologies and methodologies are shown. Relation with...

Taking into account the , chapter \ref{chap:soaea} presents a methodology to develop Service Oriented Evolutionary Algorithms (SOEAs) called SOA-EA. Steps for identification, specification, implementation and development of services are presented, with some guidelines about...

This methodology is applied in chapter \ref{chap:osgiliath} to create a framework to develop . To show the automatic binding of services a experiment... Also, two different ways of exposing services publicly are shown, and a comparison of transmission time of two different distribution technologies. Also, a comparison with 

Finally, chapter \ref{chap:conclusions} summarizes the main contributions of this thesis and future lines of research.

Figure \ref{fig:intro:piramid} summarizes the methodology applied to develop this thesis.

