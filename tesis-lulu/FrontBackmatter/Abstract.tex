%*******************************************************
% Abstract
%*******************************************************
%\renewcommand{\abstractname}{Abstract}
\pdfbookmark[1]{Resumen}{Resumen}
\begingroup
\let\clearpage\relax
\let\cleardoublepage\relax
\let\cleardoublepage\relax

\chapter*{Resumen}
Esta Tesis Doctoral pretende proporcionar un mecanismo que permita la extracci�n eficiente de las caracter�sticas presentes en una escena natural para pacientes con \textsc{Baja Visi�n}. Con esta motivaci�n, se ha desarrollado este trabajo, que presenta dos aportaciones cient�ficas principales. La primera es la adaptaci�n del operador \emph{convoluci�n} bajo el paradigma \textsc{LIP}, de tal forma que �sta permite la construcci�n de m�todos basados en filtros separables en dicho paradigma. Debido al car�cter logar�tmico de \textsc{LIP}, gracias a este operador se pueden dise�ar algoritmos de extracci�n de contornos que son invariantes ante cambios de iluminaci�n. A lo largo de la Tesis Doctoral se adaptan diversos algoritmos de extracci�n de bordes tradicionales al paradigma \textsc{LIP} utilizando el operador propuesto.\\
\noindent La segunda aportaci�n es la evaluaci�n \textsc{Subjetiva} mediante cuestionarios de la calidad de los mapas de contornos obtenidos por cada m�todo. En esta Tesis Doctoral, se proponen una serie de recomendaciones para la construcci�n de buenas encuestas para la evaluaci�n de im�genes de contornos, que han sido utilizadas para evaluar la calidad de los m�todos adaptados al paradigma \textsc{LIP} utilizando el nuevo operador propuesto. La muestra poblacional que responde a dicha encuesta incluye tanto a individuos con un grado de \textsc{Visi�n Est�ndar} como a pacientes con \textsc{Baja Visi�n}.

\vfill

\pdfbookmark[1]{Abstract}{Abstract}
\chapter*{Abstract}
This Ph. Thesis aims to provide an efficient natural scene feature extraction technique for \textsc{Low Vision} patients. This work has been developed with this motivation in mind. It provides two main scientific contributions. The first one is the adaptation of the \emph{convolution} operator to the \textsc{LIP} paradigm. It allows to adapt traditional methods based on separable filters to the \textsc{LIP} paradigm. Due to the logarithmic behaviour of \textsc{LIP}, highly invariant to illumination changes boundary extraction algorithms can be designed. Throughout this Ph. Thesis, several traditional edge extraction algorithms are adapted to \textsc{LIP} paradigm by using the new proposed operator.\\
\noindent The second contribution is the \textsc{Subjective} evaluation of the perceived quality of the contour maps obtained by each method using questionnaires. In this Ph. Thesis, several recommendations for the design of good contour images evaluation surveys have been proposed. By using these recommendations, a survey has been designed to evaluate the quality of the methods  adapted to \textsc{LIP} using the new proposed operator. The population sample that answers that survey includes both \textsc{Standard Level Vision} persons and \textsc{Low Vision} patients.


\endgroup

\vfill